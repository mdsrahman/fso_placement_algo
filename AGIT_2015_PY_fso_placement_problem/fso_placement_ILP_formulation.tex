%%This is a very basic article template.
%%There is just one section and two subsections.
\documentclass[letterpaper, 12pt]{article}
\usepackage{amsmath,amssymb}
\usepackage[left=0.7in, right= 0.7in]{geometry}
\usepackage{color}
\begin{document}
\section*{ILP Formulation}
\subsection*{Basic Equations/Notations: }Let $x_{i}$ be a node and $e_{ij}$ be
an edge (incident with nodes $x_i$ and $x_j$) in the original input network. 
\begin{equation}
	0 \leq x_i \leq 1 
\end{equation} 
\indent if $x_i$ is a feasible location (as computed from the map data) for a
base station, then $x_i$ can be 1
\begin{equation}
	0 \leq e_{ij} \leq 1 
\end{equation} 
\indent if node $x_i$ and $x_j$ are in line-of-sight (as computed
	from the map data) with each other, then $e_{ij}$ can be 1.\\
The backbone-graph is a subgraph of the above graph and is denoted by node set
$\{y_i\}$ and edge-set $\{b_{ij}\}$. From the subgraph property:
\begin{equation}
	%\forall i\mbox{ : } 
	0 \leq y_{i} \leq x_{i}
\end{equation}
\begin{equation}
	%\forall i,j\mbox{ : } 
	0 \leq b_{ij} \leq e_{ij}
\end{equation}
To enforce symmetry:
\begin{equation}
	\forall i,j \mbox{ } e_{ij} = e_{ji}
\end{equation}
\begin{equation}
	\forall i,j \mbox{ } b_{ij} = b_{ji}
\end{equation}
To enforce edge-incidence:
\begin{equation}
	e_{ij} \leq \frac{1}{2}(x_i + x_j)
\end{equation}
\begin{equation}
	b_{ij} \leq \frac{1}{2}(y_i + y_j)
\end{equation}
%Here, variables $x_i$, $y_i$, $e_{ij}$ and $b_{ij}$ have integer values 0 or 1. 
%\\ 
\subsection*{Requirements: }a) Each area is covered by at least one node in the
backbone, here $T_{ij}$ is 1 if target $j$ is covered by a node $i$:
%let this set of nodes be $T$ (we already computed this set in step
%(i) heuristic set cover algorithm). So, we want each of these nodes to be
%included in the backbone graph:
\begin{equation}
	\forall \mbox{ target }j,    \sum_{i \; \mid \; T_{ij} = 1}   y_i   \geq   1
\end{equation}
or in variable notations:
\begin{equation*}
	\forall \mbox{ target }j,    \sum_i   (T_{ij} \times y_i)   \geq   1
\end{equation*}
b) The backbone must be a connected graph with all the sink/gateway nodes in $W$
and all the source nodes in $T$. To enforce connectivity, we set up a flow problem
as follows: \\
\indent we add two nodes, a super-source node $s$ and a supersink node $t$
and make them part of the backbone node. We also add an edge from the supersource to
each source node and an edge from each sink node to the supersink:
%\begin{equation}
%	x_s = x_t = y_s = y_t = 1
%\end{equation}
%\begin{equation}
%	\forall y_i \in T \mbox{ : } e_{si} = b_{si} = 1
%\end{equation}
%\begin{equation}
%	\forall y_i \in W \mbox{ : } e_{iw} = b_{iw} = 1
%\end{equation}
\begin{equation}
	\forall i\mbox{ } 0 \leq  e_{si} \leq x_i
\end{equation}
\begin{equation}
	\forall i\mbox{ } 0 \leq b_{si} \leq y_i
\end{equation}
\begin{equation}
 	\forall w \in W\mbox{ } 0 \leq e_{wt} = 1.
\end{equation}
{\color{red}
There is a nonzero flow from $s$ to each source
node:
\begin{equation}
	\forall y_i \in T \mbox{ : }f_{si} > 0
\end{equation}
}
The total flow from $s$ must equal to that of to $t$:
\begin{equation}
	\forall i,j \mbox{ : } \sum_i f_{si} = \sum_j f_{jt}
\end{equation} 
All the flow must go through the backbone edges:
\begin{equation}
	\forall i,j \mbox{ : }f_{ij} \leq b_{ij}
\end{equation}
Also the flows must be conserved:
\begin{equation}
	\forall i,j \neq s,t \mbox{ : } \sum_i f_{ij} = \sum_i f_{ji}
\end{equation}
\subsection*{Constraints:} a) The number of node in the network graph is bounded
by a given parameter:
\begin{equation}
	\forall i \mbox{ }\sum_i x_i \leq n_{max}
\end{equation}
b) The node degree of the backbone graph is also bounded by a given parameter:
\begin{equation}
	\forall i \mbox{ } \sum_j b_{ij}  \leq d_{max}
\end{equation}
\subsection*{Objective}
Maximize the total flow from the supersource to the supersink in the network
graph. Let the flow in link $e_{ij}$ be $g_{ij}$. The flow constraint equations
can be formulated as follows:\\
The total flow from $s$ must equal to that of to $t$:
\begin{equation}
	\forall i,j \mbox{ : } \sum_i g_{si} = \sum_j g_{jt}
\end{equation}
Also the flows must be conserved:
\begin{equation}
	\forall i,j \neq s,t \mbox{ : } \sum_i g_{ij} = \sum_i g_{ji}
\end{equation}
{\color{red}
The link capacity must not be exceeded:
\begin{equation}
	\forall i,j \mbox{ } g_{ij} \leq e_{ij} \mbox{ ?? }
\end{equation}
}
The objective function is:
\begin{equation}
	\mathbf{max} \mbox{ : }\sum_i g_{si}
\end{equation}
\end{document}
